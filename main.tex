\documentclass[10pt,twocolumn,letterpaper]{article}
%% Welcome to Overleaf!
%% If this is your first time using LaTeX, it might be worth going through this brief presentation:
%% https://www.overleaf.com/latex/learn/free-online-introduction-to-latex-part-1

%% Researchers have been using LaTeX for decades to typeset their papers, producing beautiful, crisp documents in the process. By learning LaTeX, you are effectively following in their footsteps, and learning a highly valuable skill!

%% The \usepackage commands below can be thought of as analogous to importing libraries into Python, for instance. We've pre-formatted this for you, so you can skip right ahead to the title below.

\usepackage{cite}  % Este paquete es útil para comprimir múltiples citas en una sola entrada
\usepackage{indentfirst}
\usepackage[figurename=Fig.]{caption} % Cambiar el nombre del caption de las fotos
\usepackage{fancyhdr} % Definir el estilo de la página

%% Language and font encodings
\usepackage[spanish, english]{babel}
\usepackage[utf8]{inputenc}
\usepackage[T1]{fontenc}
\usepackage{times} % Times New Roman
\usepackage{courier} % Courier

%% Sets page size and margins
\usepackage[a4paper,top=1.9cm,bottom=2.54cm,left=1.73cm,right=1.73cm,marginparwidth=1.75cm]{geometry}

%% Useful packages
\usepackage{amsmath}
\usepackage{graphicx} % Images support
\usepackage[most]{tcolorbox} % Callouts Support 
\graphicspath{ {img/} }
\usepackage[colorinlistoftodos]{todonotes}
\usepackage[hidelinks]{hyperref} % Gestión de hipervínculos

\usepackage{titlesec}
\usepackage{enumitem}

%% Variables

%% Main Images
\newcommand{\logoUdg}{logo-udg.jpg}
\newcommand{\logoCucei}{logo-cucei.jpg}

% Datos de la Escuela
\newcommand{\universidad}{Universidad de Guadalajara}
\newcommand{\cede}{Centro Universitario de Ciencias Exactas e Ingenierías}

% Datos de la Materia
\newcommand{\materia}{Análisis de Algoritmos}
\newcommand{\carrera}{Ingeniería en Computación}
\newcommand{\division}{División de Tecnologías para la Integración CiberHumana}
\newcommand{\theTitle}{Problema del Erutamiento del Vehículo}
\newcommand{\profesor}{Jorge Ernesto López Arce Delgado}
\newcommand{\seccion}{D01}
\newcommand{\startDate}{21 de mayo de 2024}

% Datos de los Autores
\newcommand{\theAuthor}{Juárez Rubio Alan Yahir}
\newcommand{\bAuthor}{García González Luis Ángel}

\newcommand{\theAuthorCode}{218517809}
\newcommand{\bAuthorCode}{218557703}

\newcommand{\theAuthorMail}{alan.juarez5178@alumnos.udg.mx}
\newcommand{\bAuthorMail}{luis.garcia5577@alumnos.udg.mx}

% Datos del Repositorio
\newcommand{\repositorio}{\url{https://github.com/Forajido24/ProyectoFinal}}
\newcommand{\version}{1.0}
\newcommand{\licencia}{Sin Licencia}

% Espaciado
\newcommand{\nl}{\par\vspace{0.4cm}}

\title{\fontsize{24}{28.8}\selectfont \theTitle}
\date{}

\addto\captionsspanish{\renewcommand{\contentsname}{Índice}}

% Esilo de enlaces
\hypersetup{
	colorlinks=false
}

\urlstyle{same}

%% Config

%% Sections Formats
% section
\titleformat{\section}[block]{\centering\scshape\normalsize}{\Roman{section}.}{1em}{}
\titlespacing*{\section}{0pt}{3.5ex plus 1ex minus .2ex}{2.3ex plus .2ex}

% subsection
\titleformat{\subsection}[block]{\itshape\normalsize}{\Alph{subsection}.}{1em}{}[\hspace{1em}]
\titlespacing*{\subsection}{0pt}{3.25ex plus 1ex minus .2ex}{1em}

% subsubsection
\titleformat{\subsubsection}[runin]{\itshape\normalsize}{\arabic{subsubsection})}{1em}{}[\hspace{1em}]
\titlespacing*{\subsubsection}{0pt}{3.25ex plus 1ex minus .2ex}{1em}

%% Title
\usepackage{authblk}

\author{\theAuthor}
\author{\bAuthor}

\affil{\small{\textit{CENTRO UNIVERSITARIO DE CIENCIAS}\\
	\textit{EXACTAS E INGENIERÍAS, (CUCEI, UDG)}}}

	\affil{
		\fontfamily{courier}\selectfont
		\theAuthorMail\\
		\fontfamily{courier}\selectfont
		\bAuthorMail
	}

	\pagestyle{fancy}

% Encabezado

\renewcommand{\headrulewidth}{0pt} % Elimina la línea del encabezado
\setlength{\headheight}{15pt} % Ajuste necesario para evitar warnings

\fancyhead{}
\rhead{
	\begin{flushright}
		\theTitle
	\end{flushright}
}

\fancyfoot{}
\lfoot{\materia}
\cfoot{\thepage} % Paginación
\rfoot{\small Curso impartido por \profesor}

\begin{document}
\selectlanguage{spanish}

\begin{titlepage}
	\centering
	{\huge\textbf{\universidad}}\par\vspace{0.6cm}
	{\LARGE{\cede}}\vfill
	
	\begin{figure}[h]
		\begin{minipage}[t]{0.45\textwidth}
			\centering
			\includegraphics[width=130px, height=200px, keepaspectratio]{\logoUdg}
		\end{minipage}
		\hfill
		\begin{minipage}[t]{0.45\textwidth}
			\centering
			\includegraphics[width=130px, height=200px, keepaspectratio]{\logoCucei}
		\end{minipage}
	\end{figure}\vfill
	
	\Large{
		\division\nl
		\carrera\nl
		\textbf{\materia}\nl
	}
	\begin{figure}[h]
		\centering
		\begin{minipage}[t]{0.75\textwidth}
			{\Large
				\textbf{Integrantes:}
				\begin{itemize}
					\item \theAuthor\ - \theAuthorCode
					\item \bAuthor\ - \bAuthorCode
				\end{itemize}
			}
		\end{minipage}
	\end{figure}\vfill
	{\LARGE{\textbf{\theTitle}}}\vfill
	
	\begin{tcolorbox}[colback=red!5!white, colframe=red!75!black]
		\centering
		Este documento contiene información sensible.\\
		No debería ser impreso o compartido con terceras entidades.
	\end{tcolorbox}\vfill

	{\large \startDate}\par
\end{titlepage}

\maketitle

\noindent\textbf{\textit{Abstract}— \\
	\\
	\textit{Palabras claves} - \\
	\\
	Repositorio de código: \repositorio\\
	Versión actual del código: \version\\
	Licencia legal código: \licencia
}

\section{Introducción}
\section{Trabajos Relacionados}
\section{Descripción del Desarrollo del Proyecto Modular}
\section*{Módulo I Justificación de Arquitectura y Programación de Sistemas}
\section*{Módulo II Justificación de Sistemas Inteligentes}
\section*{Módulo III Justificación de Sistemas Distribuidos}
\section{Resultados Obtenidos del Proyecto}
\section{Conclusiones y Trabajo a Futuro}

% Referencias

\nocite{*} % Para incluir todas las referencias sin necesidad de citarlas

\clearpage
\bibliographystyle{ieeetr}
\bibliography{ref}

\end{document}
